\pdfminorversion=4 
\documentclass[conference]{IEEEtran}
\IEEEoverridecommandlockouts
% The preceding line is only needed to identify funding in the first footnote. If that is unneeded, please comment it out.
\usepackage{cite}
\usepackage{amsmath,amssymb,amsfonts}
\usepackage{algorithmic}
\usepackage{graphicx}
\usepackage{textcomp}
\usepackage{xcolor}

\usepackage{comment}
\newcommand{\cSetup}{\textit{Setup}}
\newcommand{\KeyGenServer}{\textit{KeyGenServer}}
\newcommand{\KeyGenDoctor}{\textit{KeyGenDoctor}}
\newcommand{\KeyGenPatient}{\textit{KeyGenPatient}}
\newcommand{\CreateIndex}{\textit{CreateIndex}}
\newcommand{\EncryptDocuments}{\textit{EncryptDocuments}}
\newcommand{\CreateTrapdoor}{\textit{CreateTrapdoor}}
\newcommand{\CheckAccessPolicy}{\textit{CheckAccessPolicy}}
\newcommand{\Match}{\textit{Match}}


\usepackage[ruled,linesnumbered]{algorithm2e}
\usepackage{soul,color}
\usepackage[utf8]{inputenc}
\usepackage[english]{babel}

\usepackage{subfig}
\usepackage[demo]{graphicx}


\usepackage{subfig}

\usepackage{array}
\usepackage{booktabs}
\usepackage{textcomp}
\usepackage{multirow}
\usepackage{amsmath}
\usepackage{amsthm}
\usepackage{amssymb}
\usepackage{graphicx}
\usepackage{lipsum,graphicx,multicol}
\usepackage[shortlabels]{enumitem}
\usepackage{color}

\usepackage[ruled,linesnumbered]{algorithm2e}
\definecolor{seagreen}{rgb}{0.18, 0.55, 0.24}
\SetAlFnt{\small\color{black}\tt}
\LinesNumbered
\renewcommand{\DataSty}[1]{{\color{yellow}\texttt{#1}}}
\renewcommand{\KwSty}[1]{{\color{blue}\textbf{#1}}}
\renewcommand{\ProgSty}[1]{{\color{black}\texttt{#1}}}
\renewcommand{\FuncSty}[1]{{\color{black}\texttt{#1}}}
\renewcommand{\CommentSty}[1]{{\color{seagreen}\texttt{#1}}}
\renewcommand{\ArgSty}[1]{{\color{black}\texttt{#1}}}
\renewcommand{\FuncArgSty}[1]{{\color{black}\texttt{#1}}}
\renewcommand{\ProcArgSty}[1]{{\color{black}\texttt{#1}}}
\renewcommand{\ProcNameSty}[1]{{\color{black}\texttt{#1}}}
\renewcommand{\NlSty}[1]{{\color{black}\texttt{#1}}}
%%%%%%%%%%%%%%%%%%%%%%%%%%%%%%%%%%%%%%%%%%%%%%%%%%%%%%%%%%%%%%%%%%%%%%%%

\setlength{\columnsep}{0.2 in}
\def\BibTeX{{\rm B\kern-.05em{\sc i\kern-.025em b}\kern-.08em T\kern-.1667em\lower.7ex\hbox{E}\kern-.125emX}}

\newcounter{defcounter}
\setcounter{defcounter}{0}
\usepackage{caption}
\usepackage[shortlabels]{enumitem}
\usepackage{cite}
\usepackage[ruled]{algorithm2e}
\mathchardef\period=\mathcode`.
\DeclareMathSymbol{.}{\mathord}{letters}{"3B}
\usepackage{tikz}
\tikzstyle{io} = [fill=black,inner sep=2pt,circle]
\newcommand{\hilight}[1]{\colorbox{yellow}{#1}}

\makeatletter
\newcommand*\bigcdot{\mathpalette\bigcdot@{.5}}
\newcommand*\bigcdot@[2]{\mathbin{\vcenter{\hbox{\scalebox{#2}{$\m@th#1\bullet$}}}}}
\makeatother



\makeatletter


\newtheorem{theorem}{Theorem}

\def\BibTeX{{\rm B\kern-.05em{\sc i\kern-.025em b}\kern-.08em
    T\kern-.1667em\lower.7ex\hbox{E}\kern-.125emX}}
\begin{document}



\title{Proof of Authority Consensus in SmartGrid Applications}
\author{
 \IEEEauthorblockN{John-Anthony G. Thevos~\IEEEauthorrefmark{1}, Mohamed~Baza~PhD.\IEEEauthorrefmark{1}}
  \IEEEauthorblockA{
  \IEEEauthorrefmark{1}Department of Computer Science, College of Charleston, Charleston, SC, USA\\
  }}

		


% \author{
% \IEEEauthorblockN{Sherif Abdelfattah}
% \IEEEauthorblockA{\textit{Department of Electrical and Computer Engineering} \\
% \textit{Tennessee Tech University}\\
% Cookeville, TN, USA \\
% sabdelfat42@tntech.edu}
% \and
% \IEEEauthorblockN{6\textsuperscript{th} Given Name Surname}
% \IEEEauthorblockA{\textit{dept. name of organization (of Aff.)} \\
% \textit{name of organization (of Aff.)}\\
% City, Country \\
% email address or ORCID}
% }

\maketitle

\IEEEoverridecommandlockouts
\IEEEpubid{\makebox[\columnwidth]{978-0-7381-1316-6/21/\$31.00~\copyright2021 IEEE \hfill} \hspace{\columnsep}\makebox[\columnwidth]{ }}
\maketitle
\IEEEpubidadjcol

\begin{abstract}

Existing attempts at deploying a blockchain based application in the energy sector have not proven to meet the requirements
of an environment like the energy grid. Specifically, public blockchain applications running on Proof of Work 
and Proof of Stake consensus mechanisms tend to suffer from low 
throughput speeds and cannot keep up with the near real time pace of the grid. 
These limitations make them unsuitable for most applications at scale. This 
is particularly true in a SmartGrid environment where there is demand for high throughput volume but also requires 
accuracy and querability.

Proof of Authority (PoA) consensus mechanisms depoloyed on private or consortium blockchains show promise in
bringing the throughput of blockchain applications to an acceptable level. 

This paper explores this subset of 
blockchain applications, and attempts to... wtf am I doing. 


\end{abstract}

\begin{IEEEkeywords}
SmartGrid, Blockchain, Smart contract, Proof of Authority, consortium blockchain, private blockchain 
    \end{IEEEkeywords}





\section{Introduction}
\label{introduction}

\section{Motivation}
\label{motivation}

\section{Proof of Authority}
\label{proof_of_authority}

\section{Is this just a bureaucracy?}
\label{bureacracy}

\section{Future Research}
\label{future_research}

\section{Conclusion}
\label{conclusion}

  

\bibliographystyle{IEEEtran}
\bibliography{references} 





\end{document}